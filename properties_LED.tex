\documentclass[12pt]{article}
\usepackage{amsmath}
\usepackage{graphicx}
\usepackage{physics}
\usepackage{cite}

\title{Quantum Mechanical Properties of Light Emitting Diodes}
\author{Dynamite}
\date{\today}

\begin{document}
\maketitle

\section{Introduction}
Light emitting diodes (LEDs) are quintessential optoelectronic devices that exploit the quantum mechanical properties of semiconductors. Their operation is fundamentally tied to the band structure of the materials used. By engineering band gaps, heterostructures, and quantum wells, LEDs achieve efficient radiative recombination of carriers. Organic LEDs (OLEDs) extend this principle to molecular orbitals, where recombination occurs between discrete HOMO--LUMO levels.

\section{Band Structure and Radiative Transitions}


\section{Conclusion}
The properties of LEDs and OLEDs are deeply rooted in quantum mechanics. Band structure engineering, heterostructures, and quantum wells enable efficient carrier confinement and radiative recombination. The invention of InGaN-based blue LEDs exemplifies how mastery of quantum mechanics and material science can revolutionize technology, culminating in efficient white LEDs and OLED displays.

\bibliographystyle{ieeetr}
\bibliography{led_refs}

\end{document}
